\documentclass[]{report}

\PassOptionsToPackage{hyphens}{url}\usepackage{hyperref}

\usepackage{pythonhighlight}

\usepackage{listings}

\usepackage{graphicx}

\sloppy

\usepackage{xcolor}

\definecolor{codegreen}{rgb}{0,0.6,0}
\definecolor{codegray}{rgb}{0.5,0.5,0.5}
\definecolor{codepurple}{rgb}{0.58,0,0.82}
\definecolor{backcolour}{rgb}{0.95,0.95,0.92}

\lstdefinestyle{mystyle}{
	backgroundcolor=\color{backcolour},   
	commentstyle=\color{codegreen},
	keywordstyle=\color{magenta},
	numberstyle=\tiny\color{codegray},
	stringstyle=\color{codepurple},
	basicstyle=\ttfamily\footnotesize,
	breakatwhitespace=false,         
	breaklines=true,                 
	captionpos=b,                    
	keepspaces=true,                 
	numbers=left,                    
	numbersep=5pt,                  
	showspaces=false,                
	showstringspaces=false,
	showtabs=false,                  
	tabsize=2
}

\lstset{style=mystyle}

% Title Page
\title{PyBaMM machine learning parameter data parsing - Documentation}
\author{Elias Hohl}

\begin{document}
\maketitle

\begin{abstract}
	This documents explains how to supply PyBaMM with measured parameter data dependent of one or two variables.
\end{abstract}

\section{Usage}

\subsection{Parsing 1D data}

PyBaMM can parse 1D data using interpolation. A two-column CSV file called \textit{graphite\_volume\_change\_Ai2020.csv} can be supplied as a parameter as follows:

\begin{lstlisting}[language=python]
	parameter_values = pybamm.ParameterValues()
	parameter_values['Negative electrode volume change'] = "[data]graphite_volume_change_Ai2020"
\end{lstlisting}

The columns of the CSV file should be formatted like this:

\begin{python}
independent variable 1, dependent variable
\end{python}

\subsection{Parsing 2D data}

PyBaMM was extended to parse 2D data using machine learning. Follow these steps to read data from a CSV file called \textit{lico2\_diffusivity\_Dualfoil1998.csv}:

\begin{enumerate}
\item Launch the script \textit{csv\_model\_trainer.py}. Supply the CSV path and leave all other settings at the default values. This will create a .pkl file containing the model, and a .json file containing the configuration.
\item Supply the data to PyBaMM in a similar way as done for 1D data:
	\begin{lstlisting}[language=python]
		parameter_values = pybamm.ParameterValues()
		parameter_values['Negative electrode volume change'] = "[ml data]lico2_diffusivity_Dualfoil1998"
	\end{lstlisting}
\end{enumerate}

The columns of the CSV file should be formatted like this:

\begin{python}
independent variable 1, independent variable 2, dependent variable
\end{python}

\section{Working principles}

Using the casadi 1D interpolant is pretty straightforward. However, this is not the case for 2D data. The 2D interpolant requires a perfect grid of measurement data, which is experimentally as good as impossible to generate. Therefore, a machine learning method (Random Forest Regressor) is used in a preprocessing step to create a model from the supplied data, which can be used to evaluate a grid of input data for the casadi 2D interpolant:

\begin{python}
regressor = RandomForestRegressor(n_estimators=100, random_state=0)

regressor.fit(input_data[:, 0:-1], input_data[:, -1])
\end{python}

\begin{python}
X = list(np.meshgrid(*x_))

x = np.column_stack([el.reshape(-1, 1) for el in X])

y = self.model.predict(x)
\end{python}


\end{document}